\documentclass{classrep}
\usepackage[utf8]{inputenc}
\usepackage{color}
\usepackage{ltablex}
\usepackage{makebox}
\usepackage{amsmath}
\usepackage{hyperref}

\studycycle{Informatyka, studia dzienne, II st.}
\coursesemester{II}

\coursename{Komputerowe systemy rozpoznawania}
\courseyear{2011/2012}

\courseteacher{dr inż. Arkadiusz Tomczyk}
\coursegroup{poniedziałek, 10:15}

\author{
  \studentinfo{Michał Janiszewski}{169485} \and
  \studentinfo{Mariusz Łucka}{169493}
}

\title{Zadanie Numer 2: Lingwistyczne podsumowania baz danych}

\begin{document}
\maketitle

\section{Cel}
Celem zadania było napisanie aplikacji generującej podsumowania lingwistyczne wybranej bazy danych i przedstawiającej użytkownikowi najlepsze wyniki według zastosowanych miar.

Użytkownik powinien mieć możliwość definiowania kwantyfikatorów, sumaryzatorów i kwalifikatorów oraz możliwość wyboru podsumowywanych atrybutów oraz miar wykorzystywanych do obliczenia jakości podsumowania.

\paragraph{}
Realizacja zadania składała się z 3 etapów:  
\begin{enumerate}
\item Wybór bazy danych lub wygenerowanie z niej widoku posiadającego przynajmniej 10 tysięcy rekordów oraz co najmniej 10 atrybutów dających się podsumować.

\item Implementacja biblioteki obiektowej zawierającej zbiór klas reprezentujących różne rodzaje zbiorów rozmytych typu 1 i 2 i operacji na tych zbiorach. 

\item Generowanie podsumowań dla różnych kwantyfikatorów, kwalifikatorów i sumaryzatorów.
\end{enumerate}


\section{Wprowadzenie}
Poniższy rozdział zawiera wyjaśnienie zagadnień teoretycznych i podstawowych pojęć związanych z wykonywanym zadaniem.

\subsection{Zbiory rozmyte}
Zbiorem rozmytym (ang. \textit{fuzzy set}) $A$ w przestrzeni rozważań $X$ nazywamy zbiór uporządkowanych par postaci:
\begin{equation}
A=\{<x, \mu_A(x)>:x \in X \}
\end{equation}
gdzie $\mu_A:X \rightarrow [0,1] $ jest \textit{funkcją przynależności} , a jej wartość dla elementu $x \in X$ jest \textit{stopniem przynależności} $x$ do zbioru $A$.
\paragraph{}
Uogólnieniem zbiorów rozmytych są zbiory rozmyte typu 2, których podstawą jest rozszerzenie funkcji przynależności o wartościach rzeczywistych do funkcji przynależności typu 2. Przypisuje ona każdemu elementowi z przestrzeni rozważań $X$ zbiór rozmyty typu 1.

Zbiór rozmyty typu 2 jest zdefiniowany następująco:
\begin{equation}
\tilde{A}=\{<x, \mu_{\tilde{A}}(x)>:x \in X \}
\end{equation}
gdzie $\mu_{\tilde{A}}:X \rightarrow F[0,1] $ i $F[0,1]$ jest zbiorem wszystkich zbiorów rozmytych typu 1.

\subsection{Normy trójkątne}
Normy trójkątne są uogólnieniem operacji mnożenia i iloczynu zbirów rozmytych. Wyróżniamy 2 rodzaje norm trójkątnych:

\begin{enumerate}
\item t-norma - funkcja 2 zmiennych jest t-normą jeżeli jest:
\begin{itemize}
\item przemienna,
\item łączna,
\item monotoniczna,
\item istnieje element neutralny 1;
\end{itemize}
\item s-norma (t-conorma) - funkcja 2 zmiennych jest s-normą jeżeli jest:
\begin{itemize}
\item przemienna,
\item łączna,
\item monotoniczna,
\item istnieje element neutralny 1;
\end{itemize}
\end{enumerate}

\subsection{Podstawowe operacje na zbiorach rozmytych}
Dopełnieniem $A^C$ zbioru rozmytego $A$ nazywamy zbiór rozmyty o funkcji przynależności:
\begin{equation}
\mu_{A^C}(x)=1-\mu_A(x)
\end{equation}

Sumą zbiorów $A \cup B$ rozmytych nazywamy zbiór rozmyty o funkcji przynależności:
\begin{equation}
\mu_{A \cup B}(x)=\mu_A(x)s\mu_A(x)
\end{equation}
gdzie $s$ jest dowolną s-normą np. operacją maksimum.
\\

Iloczynem zbiorów $A \cap B$ rozmytych nazywamy zbiór rozmyty o funkcji przynależności:
\begin{equation}
\mu_{A \cap B}(x)=\mu_A(x)t\mu_A(x)
\end{equation}
gdzie $t$ jest dowolną t-normą np. operacją minimum.
\\

Dopełnieniem $\tilde{A}^C$ zbioru rozmytego typu 2 $\tilde{A}$ nazywamy zbiór rozmyty typu 2 o funkcji przynależności:
\begin{equation}
\mu_{\tilde{A}^C}(x)=\int_{\mu_{\tilde{A}} \in J_x} \mu_x(\mu_{\tilde{A}})/(1-\mu_{\tilde{A}})
 \end{equation}
gdzie $\mu_{\tilde{A}^C}$ jest pierwszorzędnym stopniem przynależności elementu $x$ do zbioru $\tilde{A}$, a $\mu_x(\mu_{\tilde{A}})$ jest drugorzędnym stopniem przynależności.
\\

Suma $\tilde{A} \cup \tilde{B}$ zbiorów rozmytych typu 2 jest zdefiniowana w warunkach operacji złączenia i jest opisana funkcją przynależności:
\begin{equation}
\mu_{\tilde{A} \cup \tilde{B}}(x) = \mu_{\tilde{A}} \sqcup \mu_{\tilde{A}} = \int_{\mu_{\tilde{A}}}\int_{\mu_{\tilde{B}}}(\mu_x(\mu_{\tilde{A}}) t_1 \mu_x(\mu_{\tilde{B}}))/(\mu_{\tilde{A}} s \mu_{\tilde{B}})
\end{equation}
a iloczyn $\tilde{A} \cap \tilde{B}$ jest zdefiniowany w warunkach operacji spotkania i opisany funkcją przynależności:
\begin{equation}
\mu_{\tilde{A} \cap \tilde{B}}(x) = \mu_{\tilde{A}} \sqcap \mu_{\tilde{A}} = \int_{\mu_{\tilde{A}}}\int_{\mu_{\tilde{B}}}(\mu_x(\mu_{\tilde{A}}) t_1 \mu_x(\mu_{\tilde{B}}))/(\mu_{\tilde{A}} t_2 \mu_{\tilde{B}})
\end{equation}
gdzie $s$ jest dowolną s-normą, a $t_1$ i $t_2$ są dowolnymi t-normami

\subsection{Miary zbiorów rozmytych}
\subsubsection{Kardynalność}
Kardynalność jest podstawową miarą zbiorów rozmytych.

Dla policzalnego zbioru rozmytego A jest to suma stopni przynależności wszystkich elementów z przestrzeni rozważań:
\begin{equation}
|A|=\sum_{x \in X} \mu_A(x)
\end{equation}

Dla zbioru niepoliczalnego jest to całka:
\begin{equation}
|A|=clm(A)=\int_{x \in X} \mu_A(x)\,dx
\end{equation}
\\

W przypadku zbiorów rozmytych typu 2 wzory na kardynalności przedstawiają się następująco:
\\
dla zbioru policzalnego:
\begin{equation}
|\tilde{A}|=\sum_{x \in X} sup\{u \in J_x:\mu_x(u)=1\}
\end{equation}
i niepoliczalnego:
\begin{equation}
|\tilde{A}|=\int_{x \in X} sup\{u \in J_x:\mu_x(u)=1\}\,dx
\end{equation}

\subsubsection{Nośnik}
Nośnikiem $supp(A)$ zbioru rozmytego A jest zbiór klasyczny zawierający  
\\

Dla zbioru rozmytego typu 2 nośnikiem jest zbiór rozmyty typu 1, który zawiera elementy z przestrzeni rozważań $X$ z przypisanymi pierwszorzędnymi funkcjami przynależności, dla których drugorzędne funkcje przynależności są największe.

\subsubsection{Stopień rozmycia}
Stopień rozmycia $in(A)$ zbioru rozmytego A jest określany jako:
\begin{equation}
in(A)=\frac{|supp(A)|}{|X|}
\end{equation}
Im większa jest ta miara, tym zbiór jest bardziej nieprecyzyjny.

\subsection{Lingwistyczne podsumowania baz danych}
\subsubsection{Zmienna lingwistyczna}
Zmienna lingwistyczna $L$ to piątka $<\alpha, H, X, G, K>$, gdzie:
\begin{itemize}
\item $\alpha$ - nazwa zmiennej lingwistycznej
\item $H$ - zbiór wartości lingwistycznych, które należy podać żeby zdefiniować zmienną lingwistyczną - wartości te nazywamy etykietami
\item $X$ - przestrzeń rozważań zbiorów rozmytych, które będą skojarzone z etykietami w zbiorze $H$
\item $G$ - reguła gramatyczna, która pozwala generować etykiety. Najprostszym sposobem jest tutaj wymienienie etykiet
\item $K$ - reguła semantyczna (znaczeniowa) - mówi o tym, że dana etykietę można powiązać z danym zbiorem rozmytym
\end{itemize}

\subsubsection{Podsumowania lingwistyczne według Yagera}
Yager zaproponował podsumowania lingwistyczne następującej postaci:
\begin{itemize}
\item w formie pierwszej: QP jest/ma S [T]
\item w formie drugiej: QP będących/mających W jest/ma S [T]
\end{itemize}
gdzie:
\begin{itemize}
\item P - jest podmiotem podsumowania, czyli obiektem reprezentującym rekord w bazie danych
\item Q - jest to tzw. kwantyfikator i odnosi się do ilości rekordów, których dotyczy podsumowanie. Z kwantyfikatorem jest związana pewna wartość zmiennej lingwistycznej opisana za pomocą zbioru rozmytego. Jeżeli elementy tego zbioru mają wartości z przedziału [0,1] to kwantyfikator nazywamy względnym (relatywnym), a w przeciwnym wypadku bezwzględnym (absolutnym).
\item S i W nazywamy odpowiednio sumaryzatorem i kwalifikatorem. Są one powiązane z wartościami zmiennych lingwistycznych opisujących poszczególne kolumny bazy danych. Kwalifikator i sumaryzator mogą być również złożone z wielu wartości zmiennych lingwistycznych, wtedy są opisane za pomocą iloczynu zbiorów rozmytych związanych z poszczególnymi zmiennymi lingwistycznymi.
\item T - jest miarą prawdziwości podsumowania omówioną w kolejnej sekcji.
\end{itemize}

\paragraph{}
Podsumowania lingwistyczne pozwalają wygenerować za pomocą komputera opis zawartych w bazie danych informacji i przedstawić go w języku naturalnym. Należy jednak najpierw samodzielnie zdefiniować zmienne lingwistyczne i zbiory rozmyte odpowiadające poszczególnym kolumnom w bazie. 

\subsection{Miary jakości podsumowań lingwistycznych}
Miary jakości pozwalają z wszystkich wygenerowanych podsumowań wybrać te, które najlepiej opisują dane zawarte w bazie.

Można wyróżnić między innymi miary opisane poniżej.
\subsubsection{Stopień prawdziwości}
W przypadku kwantyfikatora względnego:
\begin{equation}
T_1 =\mu_Q(\frac{r}{m})
\end{equation}
W przypadku kwantyfikatora absolutnego:
\begin{equation}
T_1 =\mu_Q(r)
\end{equation}

Dla podsumowań w formie pierwszej:
$$r=\sum_{i=1}^m \mu_s(d_i)$$
Dla podsumowań w formie drugiej:
$$r=\frac{\sum_{i=1}^m \mu_s(d_i) \wedge \sum_{i=1}^m \mu_w(d_i)}{\sum_i=1^m \mu_s(d_i)}$$
gdzie:
$\mu_Q, \mu_W, \mu_S$ to funkcje przynależności odpowiednio: kwantyfikatora, kwalifikatora i sumaryzatora, a $d_i$ to i-ty rekord w bazie danych. 

\subsubsection{Stopień nieprecyzyjności}
\begin{equation}
T_2=(\prod_{j=1}^n in(S_j))^{\frac{1}{n}}
\end{equation}
gdzie $S_j$ jest j-tym sumaryzatorem wchodzącym w skład n-elementowego sumaryzatora złożonego.  

\subsubsection{Stopień pokrycia}
\begin{equation}
T_3=\frac{\sum_{i=1}^m t_i}{\sum_{i=1}^m h_i}
\end{equation}
gdzie:\\
\begin{equation}
t_i = \left\{
\begin{tabular}{l l}
1 & \quad \mbox{jeżeli $\mu_s(d_i) > 0$ i $\mu_s(d_i) > 0$}\\
 0 & \quad \mbox{w przeciwnym wypadku}\\ 
\end{tabular} \right .
\end{equation}

\begin{equation}
h_i = \left\{
\begin{tabular}{l l}
1 & \quad \mbox{jeżeli $\mu_s(d_i) > 0$}\\
 0 & \quad \mbox{w przeciwnym wypadku}\\ 
\end{tabular} \right .
\end{equation}

\subsubsection{Miara trafności}


\begin{equation}
T_4=|\prod_{j=1}^n r_j -T_3|
\end{equation}

\begin{equation}
r_j=\frac{\sum_{i=1}^m g_{ij}}{m}
\end{equation}

\begin{equation}
g_{ij} = \left\{
\begin{tabular}{l l}
1 & \quad \mbox{jeżeli $\mu_{s_j}(d_i) > 0$}\\
 0 & \quad \mbox{w przeciwnym wypadku}\\ 
\end{tabular} \right .
\end{equation}
gdzie $i$ jest indeksem rekordu w m-elementowej bazie, a $j$ jest indeksem sumaryzatora wchodzącego w skład n-elementowego sumaryzatora złożonego.

\subsubsection{Długość podsumowania}
\begin{equation}
T_5=2 \cdot (\frac{1}{2})^{|S|}
\end{equation}

gdzie $|S|$ jest liczbą sumaryzatorów wchodzących w skład sumaryzatora złożonego.

\subsubsection{Stopień nieprecyzyjności kwantyfikatora}
\begin{equation}
T_6=1-in(Q)
\end{equation}

\subsubsection{Stopień kardynalności kwantyfikatora}
\begin{equation}
T_7=1-(\frac{|Q|}{|X_Q|})
\end{equation}
gdzie $X_Q$ to przestrzeń rozważań kwantyfikatora $Q$.

\subsubsection{Stopień kardynalności sumaryzatora}
\begin{equation}
T_8=1-|\prod_{j=1}^n\frac{|S_j|}{X_j}|^{\frac{1}{n}}
\end{equation}
gdzie $X_j$ jest przestrzenią rozważań sumaryzatora $S_j$ wchodzącego w skład n-elementowego sumaryzatora złożonego.

\subsubsection{Stopień nieprecyzyjności kwalifikatora}
\begin{equation}
T_9=(\prod_{j=1}^x in(W_j))^{\frac{1}{x}}
\end{equation}
gdzie $W_j$ jest j-tym kwalifikatorem wchodzącym w skład x-elementowego kwalifikatora złożonego.  

\subsubsection{Stopień kardynalności kwalifikatora}
\begin{equation}
T_8=1-|\prod_{j=1}^n\frac{|S_j|}{X_j}|^{\frac{1}{n}}
\end{equation}
gdzie $X_{W_j}$ jest przestrzenią rozważań kwalifikatora $W_j$ wchodzącego w skład x-elementowego kwalifikatora złożonego.

\subsubsection{Długość kwalifikatora}
\begin{equation}
T_11=2 \cdot (\frac{1}{2})^{|W|}
\end{equation}
gdzie $|W|$ jest liczbą kwalifikatorów wchodzących w skład kwalifikatora złożonego.

\subsubsection{Miara optymalna}
Optymalną miarę podsumowania możemy policzyć jako średnią ważoną wyżej wymienionych miar. Najczęściej stosuje się równe wagi dla wszystkich miar, czyli otrzymujemy wtedy średnią arytmetyczną.

\section{Opis implementacji}
Aplikacja została napisana w języku C++ z użyciem frameworka Qt.\\
Składa się z dwóch głównych modułów. Pierwszy z nich stanowi implementację biblioteki klas reprezentujących zbiory rozmyte i operacje na nich oraz lingwistyczne podsumowania baz danych. Drugi moduł jest prostym graficznym interfejsem użytkownika, ale aplikację można również uruchamiać w trybie konsolowym.
\paragraph{}
\subsection{Najważniejsze klasy aplikacji}
Poniżej znajduje się opis najważniejszych klas stworzonych na potrzeby tego zadania:
\begin{enumerate}
\item \textbf{Funkcje przynależności}\\
Klasy te reprezentują funkcje charakterystyczne zbiorów i dziedziczą wspólny interfejs \verb|MembershipFunctionInterface|. Zostały zaimplementowane następujące typy funkcji przynależności:
\begin{enumerate}
\item \verb|TriangleFunction| - funkcja trójkątna,
\item \verb|TrapezoidFunction| - funkcja trapezoidalna, umożliwiające też reprezentowanie też funkcji prostokątnej, a co za tym idzie zbiorów klasycznych (nierozmytych),
\item \verb|DiscreteFunction| - funkcja przynależności przeznaczona do reprezentacji wartości dyskretnych, gdzie dla każdego elementu dziedziny należy jawnie podać jego stopień przynależności.
\end{enumerate}

\item \textbf{Zbiory rozmyte i operacje na zbiorach rozmytych.}\\
Wszystkie poniższe klasy dziedziczą po klasie abstrakcyjnej \verb|FuzzySet| reprezentującej zbiór rozmyty i metody do pobierania stopnia przynależności elementu do zbioru i liczby kardynalnej zbioru:
\begin{enumerate}
\item \verb|BasicFuzzySet| - podstawowa klasa zbioru rozmytego typu 1,
\item \verb|FuzzySetType2| - zbiór rozmyty typu 2,
\item \verb|Union|, \verb|Intersection|, \verb|Complement| - odpowiednio: suma, iloczyn i dopełnienie zbioru rozmytego,
\item \verb|Support|, \verb|AlphaCut| - odpowiednio: nośnik i alfa-przekrój zbioru rozmytego.
\end{enumerate}
\item \textbf{Podsumowania lingwistyczne.}
\begin{enumerate}
\item \verb|LingusisticValue| - wartość zmiennej lingwistycznej - reprezentuje zbiór rozmyty opisujący zmienną lingwistyczną, nazwę zmiennej lingwistycznej i numer kolumny bazy danych, do której się odnosi. Klasa ta reprezentuje w aplikacji kwalifikatory i sumaryzatory.
\item \verb|Quantifier| - kwantyfikator rozszerzający klasę \verb|LingusisticValue|
\item \verb|Summarization| - klasa reprezentująca pojedyncze podsumowania lingwistyczne, zawierająca kwantyfikator, listę sumaryzatorów i listę kwalifikatorów,
\item \verb|SummarizationGenerator| - generator wszystkich możliwych podsumowań (w postaci obiektów \verb|Summarization|) z podanych kwantyfikatorów, klasyfikatorów i sumaryzatorów. Umożliwia zapis podsumowań do pliku.
\item \verb|QualityMeasures| - klasa zawierająca metody służące do liczenia miar jakości podsumowań.
\end{enumerate}

\item \textbf{Klasy narzędziowe.}
\begin{enumerate}
\item \verb|FileParser| - służy do wczytywania następujących danych z plików tekstowych: baza danych, lista kwantyfikatorów oraz lista kwalifikatorów i sumaryzatorów. 
\end{enumerate}

\end{enumerate}

\subsection{Pliki wejściowe i wyjściowe}
Aplikacja korzysta z kilku plików wejściowych:
\begin{enumerate}
\item \verb|database.dat| - plik zawierający bazę danych do podsumowania. Pojedynczy wiersz reprezentuje jeden obiekt, którego atrybuty są oddzielone znakiem tabulacji.
\item \verb|data_description.txt| - opis poszczególnych kolumn bazy danych. Kolejne wiersze pliku odpowiadają kolejnym kolumnom bazy danych i mają następujący format:\\
\verb|typ_danych:nazwa_kolumy|\\
gdzie rozróżniane są 2 typy danych: N - wartość liczbowa, T - wartość tekstowa.
\item \verb|fuzzy_sets_lingvalues.txt| - zawiera opis wartości lingwistycznych, które mogą być kwalifikatorami i sumaryzatorami. Pojedynczy wiersz reprezentują jedną wartość lingwistyczną i ma następujący format:\\
\verb|etykieta:numer_kolumny:typ_funkcji_przynależności:parametry_funkcji|
gdzie mogą występować nastepujące typy funkcji przynależności: \verb|Triangle| - funkcja trójkątna, \verb|TRAPEZOID| - trapezoidalna, \verb|DISCRETE| - dyskretna, 
a parametry funkcji przynależności zależą od jej typu:
\begin{enumerate}
\item dla funkcji trójkątnej: \verb|a:b:c|, gdzie: a - wartość atrybutu dla początku podstawy trójkąta, b - wierzchołek trójkąta, c - koniec podstawy
\item dla funkcji trójkątnej: \verb|a:b:c:d|, gdzie a - początek dolnej podstawy, b - początek górnej podstawy, c - koniec górnej podstawy, d - koniec dolnej podstawy. Kiedy a=b i c=d, otrzymujemy funkcję prostokątną.
\item dla funkcji dyskretnej: ciąg par \verb|wartość_atrybutu:stopień_przynależności|
\end{enumerate}

\paragraph{}
Przykłady opisu wartości lingwistycznych:\\
\verb|wysoka temperatura na zewnątrz:4:TRAPEZOID:15:25:40:40|\\
\verb|północny kierunek wiatru:10:DISCRETE:N:1:NNE:0.75:NNW:0.75:NE:0.5:NW:0.5|\\

\item \verb|fuzzy_sets_quantifiers.txt| - zawiera opis kwantyfikatorów używanych do podsumowań. Format pliku jest podobny do opisu wartości lingwistycznych, ale zamiast numeru kolumny, występuje tutaj typ kwantyfikatora:\\
\verb|etykieta:typ_kwantyfikatora:typ_funkcji_przynależności:parametry_funkcji|\\
gdzie typ kwantyfikatora może przyjmować dwie wartości: \verb|RELATIVE| - względny, \verb|ABSOLUTE| - bezwzględny. 
\end{enumerate}

\paragraph{}
Poszczególne linie plików z opisem kwantyfikatorów, kwalifikatorów i sumaryzatorów można tymczasowo ukryć dla aplikacji, za pomocą znaku komentarza \verb|#| na początku linii.

\paragraph{}
Wynikiem działania aplikacji jest jeden plik wyjściowy zawierający podsumowania lingwistyczne w postaci tekstowej oraz ich miary jakości. Podsumowania są posortowane malejąco według miar jakości.

\section{Materiały i metody}
{\color{blue} 
Badania zostały przeprowadzone na dwóch zbiorach dokumentów tekstowych:\\
\begin{enumerate}
\item Zbiór dokumentów Reuters dostępny pod adresem:\\ \url{http://archive.ics.uci.edu/ml/datasets/Reuters-21578+Text+Categorization+Collection}\\\\
Dla tego zbioru testy były przeprowadzone w dwóch kategoriach:

\begin{itemize}
\item \textit{places} (etykiety: \textit{west-germany}, \textit{usa}, \textit{uk}, \textit{france}, \textit{canda}, \textit{japan}),
\item \textit{topics} (etykiety: \textit{coffee}, \textit{gold}, \textit{ship}, \textit{sugar}).\\
\end{itemize}
Wykorzystane zostały tylko artykuły, które w danej kategorii posiadają dokładnie jedną z wymienionych etykiet.\\


\item Zbiór wybranych na potrzeby tego zadania 100 krótkich opisów potraw ze strony \url{http://allrecipes.com}, z etykietami: \textit{cake} i \textit{pasta} (50 opisów dla każdej kategorii).

\end{enumerate}

\paragraph{}
We wszystkich badaniach zbiór danych został podzielony na 60\% danych uczących i 40\% danych testowych.
\paragraph{}
W każdym zbiorze tekstów zostały wyodrębnione wektory cech dla poszczególnych dokumentów metodami opisanymi w rozdziale 2, a następnie dokonano klasyfikacji za pomocą algorytmu k-NN dla $k$ z zakresu 1-100. W przypadku metody ekstrakcji opartej na liczbie wystąpień słów, jako słowa kluczowe wybrano te z zakresu 90-99\% najczęściej występujących, co dawało optymalne rezultaty.
Do badania odległości wektorów wykorzystano opisane wcześniej metryki: euklidesową, uliczną i Czebyszewa. Dla metryki euklidesowej i Czebyszewa została zastosowana normalizacja wektorów, ponieważ znacznie poprawiło to osiągane wyniki.\\
Również dla każdego zestawu tekstów policzono wartości podobieństwa ze tekstów zbioru treningowego do tekstów ze zbioru uczącego i dokonano klasyfikacji za pomocą k-NN.

Dodatkowo w celu poprawy otrzymywanych wyników została zastosowana zmodyfikowana przez nas metoda Jaccarda do określania podobieństwa tekstów: $$sim(d_1, d_2) = \sqrt{J(A ,B)*J(C, D)}$$\\
gdzie:\\
$A$ i $B$ to zbiory słów w dokumentach $d_1$ i $d_2$,\\
$C$ i $D$ to zbiory n-gramów 4-elementowych w dokumentach $d_1$ i $d_2$.
}

\section{Wyniki}
{\color{blue} 
Poniższe tabele przedstawiają otrzymane podczas badań wyniki:

%tabela 1

\begingroup
{\scriptsize  
\setlength{\LTleft}{-20cm plus -1fill}
\setlength{\LTright}{\LTleft}

\begin{longtable}{|p{1cm}|p{0.7cm}|p{0.7cm}|p{0.7cm}|p{0.7cm}|p{0.7cm}|p{0.7cm}|p{0.7cm}|p{0.7cm}|p{0.7cm}|p{0.7cm}|p{0.7cm}|p{0.7cm}|p{0.7cm}|p{0.7cm}|p{1.1cm}|}
\caption{ Procent poprawnych wyników dla metody ekstrakcji opartej na liczbie wystąpień słów i dla metryki euklidesowej.}\\ 
\hline

Zbiór
dokumentów

 &\multicolumn{15}{c|}{Parametr k}\\
\cline{2-16}
& 1
& 2
& 3
& 4
& 5
& 6
& 7
& 8
& 9
& 10
& 20
& 40
& 60
& 100
& Najlepszy wynik
\\ \hline\hline
Reuters
- places
& 87.86\%	%1
& 85.03\%	%2
& 89.56\%	%3
& 88.91\%	%4
& 89.87\%	%5
& 90.04\%	%6
& 90.41\%	%7
& 90.17\%	%8
& 90.50\%	%9
& 90.17\%	%10
& 89.36\%	%20
& 88.64\%	%30
& 88.05\%	%60
& 86.86\%	%100
& 90.17\% (k=8,10)
\\ \hline
Reuters
- topics
& 91.26\%	%1
& 86.41\%	%2
& 90.29\%	%3
& 88.35\%	%4
& 91.26\%	%5
& 91.75\%	%6
& 93.20\%	%7
& 92.72\%	%8
& 93.20\%	%9
& 92.23\%	%10
& 92.23\%	%20
& 91.26\%	%40
& 91.75\%	%60
& 92.72\%	%100
& 93.20\% (k=7,9)
\\ \hline
Przepisy
- kulinarne 
& 77.5\%	%1
& 80\%		%2
& 80\%		%3
& 77.5\%	%4
& 77.5\%	%5
& 80\%		%6
& 75\%		%7
& 77.5\%	%8
& 77.5\%	%9
& 80\%		%10
& 90\%		%20
& 77.5\%	%40
& ---		%60
& ---		%100
& 90\% (k=20)
\\ \hline
\end{longtable}
}
\endgroup


%tabela 2

\begingroup
{\scriptsize  
\setlength{\LTleft}{-20cm plus -1fill}
\setlength{\LTright}{\LTleft}

\begin{longtable}{|p{1cm}|p{0.7cm}|p{0.7cm}|p{0.7cm}|p{0.7cm}|p{0.7cm}|p{0.7cm}|p{0.7cm}|p{0.7cm}|p{0.7cm}|p{0.7cm}|p{0.7cm}|p{0.7cm}|p{0.7cm}|p{0.7cm}|p{1.1cm}|}
\caption{ Procent poprawnych wyników dla metody ekstrakcji opartej na liczbie wystąpień słów i dla metryki ulicznej.}\\ 
\hline

Zbiór
dokumentów

 &\multicolumn{15}{c|}{Parametr k}\\
\cline{2-16}
& 1
& 2
& 3
& 4
& 5
& 6
& 7
& 8
& 9
& 10
& 20
& 40
& 60
& 100
& Najlepszy wynik
\\ \hline\hline
Reuters
- places
& 85.64\%	%1
& 71.25\%	%2
& 85.48\%	%3
& 74.63\%	%4
& 85.33\%	%5
& 84.52\%	%6
& 84.83\%	%7
& 84.29\%	%8
& 84.31\%	%9
& 83.92\%	%10
& 82.65\%	%20
& 81.26\%	%40
& 81.25\%	%60
& 81.23\%	%100
& 85.64\% (k=1)
\\ \hline
Reuters
- topics
& 91.26\%	%1
& 88.35\%	%2
& 89.81\%	%3
& 91.26\%	%4
& 91.26\%	%5
& 92.72\%	%6
& 91.75\%	%7
& 92.23\%	%8
& 91.75\%	%9
& 91.75\%	%10
& 91.75\%	%20
& 90.78\%	%40
& 89.32\%	%60
& 88.34\%	%100
& 92.72\%  (k=6)
\\ \hline
Przepisy
- kulinarne 
& 62.5\%	%1
& 57.5\%	%2
& 75\%		%3
& 62.5\%	%4
& 80\%		%5
& 72.5\%	%6
& 82.5\%	%7
& 75\%		%8
& 75\%		%9
& 77.5\%	%10
& 85\%		%20
& 82.5\%	%40
& ---		%60
& ---		%100
& 92.5\% (k=17)
\\ \hline
\end{longtable}
}
\endgroup



%tabela 3

\begingroup
{\scriptsize  
\setlength{\LTleft}{-20cm plus -1fill}
\setlength{\LTright}{\LTleft}

\begin{longtable}{|p{1cm}|p{0.7cm}|p{0.7cm}|p{0.7cm}|p{0.7cm}|p{0.7cm}|p{0.7cm}|p{0.7cm}|p{0.7cm}|p{0.7cm}|p{0.7cm}|p{0.7cm}|p{0.7cm}|p{0.7cm}|p{0.7cm}|p{1.1cm}|}
\caption{ Procent poprawnych wyników dla metody ekstrakcji opartej na liczbie wystąpień słów i dla metryki Czebyszewa.}\\ 
\hline

Zbiór
dokumentów

 &\multicolumn{15}{c|}{Parametr k}\\
\cline{2-16}
& 1
& 2
& 3
& 4
& 5
& 6
& 7
& 8
& 9
& 10
& 20
& 40
& 60
& 100
& Najlepszy wynik
\\ \hline\hline
Reuters
- places
& 79.23\%	%1
& 72.71\%	%2
& 81.58\%	%3
& 81.67\%	%4
& 83.46\%	%5
& 83.70\%	%6
& 84.40\%	%7
& 84.29\%	%8
& 84.42\%	%9
& 84.52\%	%10
& 84.16\%	%20
& 83.09\%	%40
& 82.13\%	%60
& 81.41\%	%100
& 84.59\% (k=11)
\\ \hline
Reuters
- topics
& 75.73\%	%1
& 75.24\%	%2
& 74.76\%	%3
& 72.82\%	%4
& 70.87\%	%5
& 68.45\%	%6
& 68.93\%	%7
& 66.99\%	%8
& 66.99\%	%9
& 65.05\%	%10
& 57.28\%	%20
& 41.75\%	%40
& 34.95\%	%60
& 45.14\%	%100
& 75.73\%	(k=1)
\\ \hline
Przepisy
- kulinarne 
& 67.5\%	%1
& 72.5\%	%2
& 65\%		%3
& 75\%		%4
& 62.5\%	%5
& 67.5\%	%6
& 70\%		%7
& 72.5\%	%8
& 65\%		%9
& 72.5\%	%10
& 60\%		%20
& 52.5\%	%40
& ---		%60
& ---		%100
& 75\% (k=4)
\\ \hline
\end{longtable}
}
\endgroup




%tabela 4

\begingroup
{\scriptsize  
\setlength{\LTleft}{-20cm plus -1fill}
\setlength{\LTright}{\LTleft}

\begin{longtable}{|p{1cm}|p{0.7cm}|p{0.7cm}|p{0.7cm}|p{0.7cm}|p{0.7cm}|p{0.7cm}|p{0.7cm}|p{0.7cm}|p{0.7cm}|p{0.7cm}|p{0.7cm}|p{0.7cm}|p{0.7cm}|p{0.7cm}|p{1.1cm}|}
\caption{ Procent poprawnych wyników dla metody gęstości informacji i metryki euklidesowej.}\\ 
\hline

Zbiór
dokumentów

 &\multicolumn{15}{c|}{Parametr k}\\
\cline{2-16}
& 1
& 2
& 3
& 4
& 5
& 6
& 7
& 8
& 9
& 10
& 20
& 40
& 60
& 100
& Najlepszy wynik
\\ \hline\hline
Reuters
- places
& 79.18\%	%1
& 77.72\%	%2
& 79.93\%	%3
& 79.38\%	%4
& 80.27\%	%5
& 79.97\%	%6
& 80.48\%	%7
& 80.43\%	%8
& 80.88\%	%9
& 80.88\%	%10
& 80.91\%	%20
& 80.86\%	%40
& 80.86\%	%60
& 80.78\%	%100
& 81.02\% (k=24)

\\ \hline
Reuters
- topics
& 77.72\%	%1
& 77.45\%	%2
& 76.52\%	%3
& 73.52\%	%4
& 70.91\%	%5
& 69.55\%	%6
& 69.01\%	%7
& 67.12\%	%8
& 66.56\%	%9
& 66.32\%	%10
& 60.36\%	%20
& 40.65\%	%40
& 38.45\%	%60
& 47.32\%	%100
& 77.72\%	(k=1)
\\ \hline
Przepisy
- kulinarne 
& 60\%		%1
& 62.5\%	%2
& 62.5\%	%3
& 65\%		%4
& 67.5\%	%5
& 70\%		%6
& 75\%		%7
& 70\%		%8
& 70\%		%9
& 65\%		%10
& 45\%		%20
& 45\%		%40
& ---		%60
& ---		%100
& 75\% (k=7)
\\ \hline
\end{longtable}
}
\endgroup



%tabela 5

\begingroup
{\scriptsize  
\setlength{\LTleft}{-20cm plus -1fill}
\setlength{\LTright}{\LTleft}

\clearpage

\begin{longtable}{|p{1cm}|p{0.7cm}|p{0.7cm}|p{0.7cm}|p{0.7cm}|p{0.7cm}|p{0.7cm}|p{0.7cm}|p{0.7cm}|p{0.7cm}|p{0.7cm}|p{0.7cm}|p{0.7cm}|p{0.7cm}|p{0.7cm}|p{1.1cm}|}
\caption{ Procent poprawnych wyników dla pierwszej metody ekstrakcji i metryki ulicznej.}\\ 
\hline

Zbiór
dokumentów

 &\multicolumn{15}{c|}{Parametr k}\\
\cline{2-16}
& 1
& 2
& 3
& 4
& 5
& 6
& 7
& 8
& 9
& 10
& 20
& 40
& 60
& 100
& Najlepszy wynik
\\ \hline\hline
Reuters
- places
& 77.77\%	%1
& 74.43\%	%2
& 79.99\%	%3
& 79.36\%	%4
& 80.97\%	%5
& 80.61\%	%6
& 81.04\%	%7
& 81.08\%	%8
& 81.34\%	%9
& 81.67\%	%10
& 81.32\%	%20
& 80.88\%	%40
& 80.32\%	%60
& 80.16\%	%100
& 81.67\% (k=10)
\\ \hline
Reuters
- topics
& 77.03\%	%1
& 77.42\%	%2
& 75.37\%	%3
& 74.52\%	%4
& 72.42\%	%5
& 70.54\%	%6
& 69.99\%	%7
& 68.13\%	%8
& 64.69\%	%9
& 65.01\%	%10
& 60.45\%	%20
& 43.88\%	%40
& 35.56\%	%60
& 47.64\%	%100
& 77.42\%	(k=2)
\\ \hline
Przepisy
- kulinarne 
& 72.5\%	%1
& 72.5\%	%2
& 75\%		%3
& 77.5\%	%4
& 72.5\%	%5
& 70\%		%6
& 65\%		%7
& 62.5\%	%8
& 62.5\%	%9
& 65\%		%10
& 60\%		%20
& 45\%		%40
& ---		%60
& ---		%100
& 77.5\% (k=4)
\\ \hline
\end{longtable}
}
\endgroup



%tabela 6

\begingroup
{\scriptsize  
\setlength{\LTleft}{-20cm plus -1fill}
\setlength{\LTright}{\LTleft}

\begin{longtable}{|p{1cm}|p{0.7cm}|p{0.7cm}|p{0.7cm}|p{0.7cm}|p{0.7cm}|p{0.7cm}|p{0.7cm}|p{0.7cm}|p{0.7cm}|p{0.7cm}|p{0.7cm}|p{0.7cm}|p{0.7cm}|p{0.7cm}|p{1.1cm}|}
\caption{ Procent poprawnych wyników dla metody gęstości informacji i metryki Cebyszewa.}\\ 
\hline

Zbiór
dokumentów

 &\multicolumn{15}{c|}{Parametr k}\\
\cline{2-16}
& 1
& 2
& 3
& 4
& 5
& 6
& 7
& 8
& 9
& 10
& 20
& 40
& 60
& 100
& Najlepszy wynik
\\ \hline\hline
Reuters
- places
& 79.84\%	%1
& 79.84\%	%2
& 79.84\%	%3
& 79.84\%	%4
& 79.84\%	%5
& 79.84\%	%6
& 79.84\%	%7
& 79.84\%	%8
& 79.84\%	%9
& 79.84\%	%10
& 79.84\%	%20
& 79.84\%	%40
& 79.84\%	%60
& 79.84\%	%100
& 79.84\% (k=1-100)
\\ \hline
Reuters
- topics
& 60.32\%	%1
& 60.32\%	%2
& 60.32\%	%3
& 60.32\%	%4
& 60.32\%	%5
& 60.32\%	%6
& 60.32\%	%7
& 60.32\%	%8
& 60.32\%	%9
& 60.32\%	%10
& 60.32\%	%20
& 60.32\%	%40
& 60.32\%	%60
& 60.32\%	%100
& 60.32\% (k=1-100)
\\ \hline
Przepisy
- kulinarne 
& 60\%	%1
& 60\%	%2
& 60\%		%3
& 60\%	%4
& 60\%	%5
& 60\%		%6
& 60\%		%7
& 60\%	%8
& 60\%	%9
& 60\%		%10
& 60\%		%20
& 60\%		%40
& ---		%60
& ---		%100
& 60\% (k=1-100)
\\ \hline
\end{longtable}
}
\endgroup



%tabela 7

\begingroup
{\scriptsize  
\setlength{\LTleft}{-20cm plus -1fill}
\setlength{\LTright}{\LTleft}

\begin{longtable}{|p{1cm}|p{0.7cm}|p{0.7cm}|p{0.7cm}|p{0.7cm}|p{0.7cm}|p{0.7cm}|p{0.7cm}|p{0.7cm}|p{0.7cm}|p{0.7cm}|p{0.7cm}|p{0.7cm}|p{0.7cm}|p{0.7cm}|p{1.1cm}|}
\caption{ Procent poprawnych wyników dla podobieństwa tekstów za pomocą miary Jaccarda.}\\ 
\hline

Zbiór
dokumentów

 &\multicolumn{15}{c|}{Parametr k}\\
\cline{2-16}
& 1
& 2
& 3
& 4
& 5
& 6
& 7
& 8
& 9
& 10
& 20
& 40
& 60
& 100
& Najlepszy wynik
\\ \hline\hline
Reuters
- places
& 88.30\%	%1
& 85.83\%	%2
& 89.70\%	%3
& 89.67\%	%4
& 89.67\%	%5
& 89.32\%	%6
& 89.63\%	%7
& 89.63\%	%8
& 89.62\%	%9
& 89.63\%	%10
& 88.71\%	%20
& 86.83\%	%40
& 85.55\%	%60
& 83.92\%	%100
& 89.70\%	(k=3)
\\ \hline
Reuters
- topics
& 85.92\%	%1
& 82.52\%	%2
& 83.98\%	%3
& 83.50\%	%4
& 86.41\%	%5
& 87.86\%	%6
& 87.86\%	%7
& 83.98\%	%8
& 86.89\%	%9
& 86.41\%	%10
& 86.89\%	%20
& 85.44\%	%40
& 81.55\%	%60
& 82.04\%	%100
& 87.86\%	(k=6,7)
\\ \hline
Przepisy
- kulinarne 
& 85\%		%1
& 80\%		%2
& 85\%		%3
& 82.5\%	%4
& 85\%		%5
& 82.5\%	%6
& 87.5\%	%7
& 82.5\%	%8
& 90\%		%9
& 82.5\%	%10
& 85\%		%20
& 87.5\%	%40
& ---		%60
& ---		%100
& 90\% (k=9)
\\ \hline
\end{longtable}
}
\endgroup



\clearpage
%tabela 8

\begingroup
{\scriptsize  
\setlength{\LTleft}{-20cm plus -1fill}
\setlength{\LTright}{\LTleft}

\begin{longtable}{|p{1cm}|p{0.7cm}|p{0.7cm}|p{0.7cm}|p{0.7cm}|p{0.7cm}|p{0.7cm}|p{0.7cm}|p{0.7cm}|p{0.7cm}|p{0.7cm}|p{0.7cm}|p{0.7cm}|p{0.7cm}|p{0.7cm}|p{1.1cm}|}
\caption{ Procent poprawnych wyników dla podobieństwa tekstów za pomocą n-gramów 4-elementowych.}\\ 
\hline

Zbiór
dokumentów

 &\multicolumn{15}{c|}{Parametr k}\\
\cline{2-16}
& 1
& 2
& 3
& 4
& 5
& 6
& 7
& 8
& 9
& 10
& 20
& 40
& 60
& 100
& Najlepszy wynik
\\ \hline\hline
Reuters
- places
& 78.38\%	%1
& 75.96\%	%2
& 79.73\%	%3
& 79.36\%	%4
& 80.11\%	%5
& 80.14\%	%6
& 80.49\%	%7
& 80.34\%	%8
& 80.49\%	%9
& 80.38\%	%10
& 80.43\%	%20
& 80.24\%	%40
& 80.21\%	%60
& 80.24\%	%100
& 80.54\%	(k=16)
\\ \hline
Reuters
- topics
& 60.19\%	%1
& 57.77\%	%2
& 63.59\%	%3
& 61.17\%	%4
& 61.17\%	%5
& 59.71\%	%6
& 61.17\%	%7
& 59.71\%	%8
& 61.66\%	%9
& 61.66\%	%10
& 66.50\%	%20
& 71.36\%	%40
& 70.87\%	%60
& 74.27\%	%100
& 75.72\%	(k=97)
\\ \hline
Przepisy
- kulinarne 
& 77.5\%	%1
& 70\%		%2
& 82.5\%	%3
& 72.5\%	%4
& 75\%		%5
& 72.5\%	%6
& 82.5\%	%7
& 80\%		%8
& 87.5\%	%9
& 87.5\%	%10
& 82.5\%	%20
& 92.5\%	%40
& ---		%60
& ---		%100
& 95\% (k=41)
\\ \hline
\end{longtable}
}
\endgroup



%tabela 9

\begingroup
{\scriptsize  
\setlength{\LTleft}{-20cm plus -1fill}
\setlength{\LTright}{\LTleft}

\begin{longtable}{|p{1cm}|p{0.7cm}|p{0.7cm}|p{0.7cm}|p{0.7cm}|p{0.7cm}|p{0.7cm}|p{0.7cm}|p{0.7cm}|p{0.7cm}|p{0.7cm}|p{0.7cm}|p{0.7cm}|p{0.7cm}|p{0.7cm}|p{1.1cm}|}
\caption{ Procent poprawnych wyników dla podobieństwa tekstów za pomocą zmodyfikowanej miary Jaccarda.}\\ 
\hline

Zbiór
dokumentów

 &\multicolumn{15}{c|}{Parametr k}\\
\cline{2-16}
& 1
& 2
& 3
& 4
& 5
& 6
& 7
& 8
& 9
& 10
& 20
& 40
& 60
& 100
& Najlepszy wynik
\\ \hline\hline
Reuters
- places
& 88.54\%	%1
& 85.64\%	%2
& 89.39\%	%3
& 89.49\%	%4
& 90.23\%	%5
& 89.89\%	%6
& 89.78\%	%7
& 89.87\%	%8
& 89.63\%	%9
& 89.65\%	%10
& 88.34\%	%20
& 86.66\%	%40
& 85.35\%	%60
& 83.70\%	%100

& 90.23\%	(k=5)
\\ \hline
Reuters
- topics
& 83.98\%	%1
& 82.52\%	%2
& 84.46\%	%3
& 86.41\%	%4
& 84.95\%	%5
& 88.35\%	%6
& 87.38\%	%7
& 85.44\%	%8
& 85.92\%	%9
& 85.44\%	%10
& 81.55\%	%20
& 80.10\%	%40
& 81.07\%	%60
& 81.07\%	%100

& 88.35\%	(k=6)
\\ \hline
Przepisy
- kulinarne 
& 92.5\%	%1
& 82.5\%	%2
& 80\%		%3
& 80\%		%4
& 80\%		%5
& 77.5\%	%6
& 80\%		%7
& 77.5\%	%8
& 85\%		%9
& 82.5\%	%10
& 85\%		%20
& 85\%		%40
& ---		%60
& ---		%100
& 92.5\% (k=1,13)
\\ \hline
\end{longtable}
}
\endgroup

}

\section{Dyskusja}
{\color{blue} 
W metodzie $k$-NN metryka Czebyszewa daje najgorsze wyniki spośród badanych metryk, ponieważ jest bardzo wrażliwa na maksymalną różnicę współrzędnych i nawet jeżeli pozostałe współrzędne są podobne, to wektory są klasyfikowane jako odległe od siebie. Metryki euklidesowa i uliczna dają podobne, dobre rezultaty, chociaż metryka euklidesowa jest nieznacznie lepsza.

Współczynnik Jaccarda dobrze sprawdza się jako miara podobieństwa tekstów. Metoda n-gramów jest skuteczna jedynie dla tekstów gdzie we wszystkich artykułach w danej kategorii występują podobne słowa (przypadek dla przepisow kulinarnych).

Jak pokazuje zmodyfikowana przez nas miara Jaccarda, połączenie ze sobą kilku metod może przynieść poprawę wyników; do uzyskania znacznej poprawy konieczna jest jednak dogłębna znajomość dziedziny.

Metoda ekstrakcji cech tekstów oparta na liczbie wystąpień słów, pomimo że wydaje się prosta, okazała się dużo bardziej skuteczna od metody gęstości informacji

Zaimplementowane Metody nie sprawiają wrażenia złożonych obliczeniowo, jednak ilość danych, jaką muszą one przetworzyć jest znaczna. Implementacja metod jest bardzo dobrym polem do optymalizacji, niejednokrotnie zdarzało się nam przyspieszyć program o więcej niż 2x. Wciąż nie pozwoliło to jednak na policzenie wyników w rozsądnym czasie dla wszystkich testów \ppauza z części testów musieliśmy zrezygnować.
}

\section{Wnioski}
{\color{blue} 
\begin{itemize}
 \item Zmiana metryki w metodzie $k$-NN powoduje duże zmiany skuteczności.
 \item Nie ma metody uniwersalnej, skuteczność jest zależna od dobranych parametrów.
 \item Do klasyfikacji przydaje się wszelka wiedza o klasyfikowanych obiektach.
 \item Często należy ręcznie dobrać parametry metod.
\end{itemize}
}

\begin{thebibliography}{0}
\bibitem .A. Niewiadomski, \textsl{Methods for the Linguistic Summarization of Data: Applications of Fuzzy Stes and Their Extensions}, Akademicka Oficyna Wydawnicza EXIT , Warszawa 2008
	
\end{thebibliography}
\end{document}
